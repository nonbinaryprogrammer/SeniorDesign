\documentclass[letterpaper,10pt]{article}
\title{Review: Many Voices Publishing Platform}
\author{Helena~Bales\\ \\ CS463 - Spring 2017}

\usepackage[pdftex]{graphicx}
\usepackage{tikz}
\usepackage{geometry}
\usepackage{hyperref}
\usepackage{float}

\usepackage{wrapfig}
\usepackage{listings}
\usepackage{color}
\geometry{textheight=8.5in, textwidth=4in}

\parindent = 0.0 in
\parskip = 0.1 in

\begin{document}
\huge{Review: Many Voices\\ Publishing Platform}\\
\large{By Helena Bales - May 5th, 2017}\\

\normalsize{
Textbooks present a dilemma for students and professors alike. For students, the question is if, and 
how they should buy the assigned textbook. For professors, the problem is finding textbooks that have 
the content that they want and are affordable for their students. Evan Tschuy and his team present a 
solution to the problems of both students and professors with the Many Voices Publishing Platform, a 
web application for professors to create content for their classes.

The Many Voices Publishing Platform moves away from the longstanding view of textbooks as a 
proprietary product, and creates a creative and collaborative online community of professors creating 
new content and modifying existing content. The Many Voices Publishing Platform allows professors to 
build on existing work from other professors using the platform. Once they have found content similar 
to what they need for their course, they can make a copy of the project and edit the content.
}

\begin{figure}[ht]
\noindent\rule{6cm}{0.4pt}\\
\large{Helena Recommends}\\
\Large{Many Voices\\ Publishing Platform}\\
\large{8/10}\\
\noindent\rule{6cm}{0.4pt}\\

\normalsize{
\newgeometry{textwidth=6cm}
\textbf{WIRED}\\
This system encourages the sharing and growth of ideas by helping professors make custom content for 
their classes. It also provides students with a less expensive way to access their course materials, and provides them with content selected by their professors for their class. It shortens the amount of time needed between starting a textbook and getting it to the students.\\

\textbf{TIRED}\\
As the platform is still in its first release version, some more advanced features are missing, such 
as including practice problems in the book, and pulling in and citing external resources. While both 
of these tasks can still be completed, it requires further knowledge, and practice problems are 
handled in the same way as other chapters or sections of the book. This application may present some 
barrier to entry for professors as some knowledge of LaTeX is required.\\
\restoregeometry
\noindent\rule{6cm}{0.4pt}\\
}
\label{fig:Recommendation}
\end{figure}

\normalsize{
The process for professors goes like this. First, the professor searches the platform’s existing material by content, tags, authors, etc., and selects a project to fork. Once a project has been selected, the professor has the chance to edit the content, import chapters, import paragraphs, and write new content. All of this editing is done within the platform using LaTeX, a markdown language common in the academic world for writing journal articles and papers for conferences. While LaTeX may make it more difficult for new professors to use this system, it is part of what makes the Many Voices Publishing Platform powerful and versatile.

The focus on collaboration and editing allows the Many Voices Publishing Platform to provide class 
content that is more customized for a lower price. For students, the application is just another 
online textbook, except that the content is selected by the professor for their course. For the 
professor, the platform allows for continuing customization of their content with the ability to 
roll back to previous versions thanks to a version control system within the platform.

This application is still in development and more features are in the planning stages, but the Many 
Voices Publishing Platform presents its unique solution to the problem of textbooks. This 
collaborative solution requires professors to know some LaTeX, a formatting language, but it presents 
a great opportunity for the future of collaborative content development for higher learning.
}

\end{document}
