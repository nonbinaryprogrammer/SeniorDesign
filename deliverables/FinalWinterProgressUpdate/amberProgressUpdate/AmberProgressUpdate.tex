\documentclass[letterpaper,10pt]{article}
\title{Winter Progress Report for RockSat-X Payload - Hephaestus \\ Group 26}
\author{Amber~Horvath\\ \\ CS462 - Winter 2017}

\parindent = 0.0 in
\parskip = 0.1 in

\begin{document}
\maketitle

\section{Introduction}
The Hephaestus Payload is a rocketry payload that will fly onboard the 2016-2017 RockSat-X rocket. 
The rocket will be launched from the Wallops Flight Facility filled with student-made payloads from 
various institutions. The Hephaestus payload will consist of a deployable arm and a video camera.
The arm shall extend and move to a series of pre-placed sensors and make contact with the sensors. 
The arm will then contract and retract back into the rocket.
The video camera shall record the arm's movement. Data about the flight, such as error codes, shall be sent via a telemetry port and written onto a microSD card.
The Hephaestus mission will be Oregon State University's first space mission and will prove not only
our ability to develop a space-ready payload, but also the viability of construction in space using a robotic
arm.

\section{Individual Pieces}
The individual pieces of the project I am in charge of for the project include Emergency Retraction, 
Modes of Operation, and Touch Sensor Viability. I also took on the Data Storage task by the request
of the Electrical Engineers.
\subsection{Emergency Retraction}
During the past term, 

\end{document}