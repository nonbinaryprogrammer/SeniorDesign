\subsubsection{Target Generation}
TODO

\subsubsection{Arm Movement}
TODO

\subsubsection{Arm Position Tracking}
TODO

\subsubsection{Emergency Payload Expulsion}
TODO

\subsubsection{Program Modes of Operation}
TODO

\subsubsection{Target Success Sensors}
TODO

\subsubsection{Telemetry}
The design of the telemetry interface has changed multiple times over the
course of the year, mainly due to the poor understand of how telemetry systems
work on the rocket.

The design was originally going to transmit the result of a series of tests
that the \gls{payload} would perform.
However, when the arm pathing design changed from dynamically generating target 
points to using a set of predefined target points, we also changed the way the
result of these operations would be reported via telemetry.
We ultimately decided on a set of 4 digit codes to transmit through telemetry,
since we had four pins to transmit data through the rocket's telemetry system.
Each code would represent a milestone in the program, like transitioning to a
new phase or successfully touching touch sensor.

Additionally, we did not plan on storing a log file.
However, later in the design process we decided to add an SD card to the
\gls{payload} to store data on.
To take advantage of the huge amount of data storage available to us, we
decided to store a detailed log of all operations to better assist in debugging
after the flight.
However, after significant implementation difficulties, we were forced to
abandon the SD card.
Instead of aborting the log file, we changed the storage target of the logging
\gls{api} from the SD card to an on-board \gls{eeprom} chip.
This introduced a new constraint, as the \gls{eeprom} chip had only 4 kB of
storage space, as opposed to the near limitless storage capacity of the SD 
card.
We had to reduce the total amount of logging done in order to remain under
the 4 kB limit.
The logging format will remain the same.

\subsubsection{Video Camera}
When this document was first drafted, the video was originally designed to be
handled by the \gls{obc}.
However, as the electrical system was designed and finalized, it became easier
to have the video completely handled in hardware.
As a result, the \gls{obc} does not interact with the video camera beyond
turning it on at the beginning of the experiment and turning it off at the end.
The electrical system manages storing the data.

\subsubsection{Data Visualization and Processing}
The only change to the design of the data visualization and processing
component is the parsing component for the log file. 
Instead of opening an \gls{ascii} text file stored on the SD card, this
component must parse a raw binary file copied from the \gls{eeprom} chip.
