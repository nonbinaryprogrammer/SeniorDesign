\subsubsection{Target Generation}
TODO

\subsubsection{Arm Movement}
TODO

\subsubsection{Arm Position Tracking}
TODO

\subsubsection{Emergency Payload Expulsion}
The design of the emergency payload expulsion changed multiple times due to
changing requirements and understanding of how the \gls{payload} and arm itself would fit into the rocket's limited space.

Originally, the plan was to disattach the arm and eject it into space upon
encountering an issue that prevented us from being able to properly retract the arm into the body of the payload,
such as the arm getting stuck on some object. However, upon request from our contacts at the RockSat-X facility in 
Colorado, we changed the design such that, in the event of the arm getting stuck or the experiment timing out, 
we power off all motors except the motor attached to the lower deck plate, and just pull the arm in. This
implementation works with the inclusion of an interrupt vector that is triggered by the event timer
going high, signalling the end of our time in apogee or the previous modes failing to return a value indicating
that the arm is in its callibrated, home position.

\subsubsection{Program Modes of Operation}
The design for the program modes of operation hardly changed throughout the course of the project.
The only difference from our initial design and final design was the inclusion of a state representing the
retract phase. The reason for adding the retract phase is that we originally imagined retraction would only happen during the beginning
and end of the flight and thus would be included within the Observation and Off phases, respectively. However, since we changed our
emergency payload expulsion to essentially a retraction mechanism, we wanted to include this relationship within our design of the
program structure.

\subsubsection{Target Success Sensors}
The design of the target success sensors has changed multiple times over the course of the year,
mostly due to time constraints and evolving understanding of the problem and how to move a robotic arm with 4
degrees of motion.

The original design included the automatic generation of randomly selected points within the confines of the payload,
and the \gls{OBS} performing an automatic check to determine after motor movements whether it's made it to that point. While
this could be future work for later students and would allow for a more flexible design, our team did not have the capacity
to perform this dynamic moving aboard the \gls{payload}. Instead, early on within the project, we changed the design to its current state
where we have pre-placed sensors onboard the \gls{payload}. Upon a sensor being depressed, a signal is sent and a code is sent over 
the telemetry lines and a message is written to EEPROM.

\subsubsection{Telemetry}
The design of the telemetry interface has changed multiple times over the
course of the year, mainly due to the poor understanding of how telemetry systems
work on the rocket.

The design was originally going to transmit the result of a series of tests
that the \gls{payload} would perform.
However, when the arm pathing design changed from dynamically generating target 
points to using a set of predefined target points, we also changed the way the
result of these operations would be reported via telemetry.
We ultimately decided on a set of 4 digit codes to transmit through telemetry,
since we had four pins to transmit data through the rocket's telemetry system.
Each code would represent a milestone in the program, like transitioning to a
new phase or successfully touching touch sensor.

Additionally, we did not plan on storing a log file.
However, later in the design process we decided to add an SD card to the
\gls{payload} to store data on.
To take advantage of the huge amount of data storage available to us, we
decided to store a detailed log of all operations to better assist in debugging
after the flight.
However, after significant implementation difficulties, we were forced to
abandon the SD card.
Instead of aborting the log file, we changed the storage target of the logging
\gls{api} from the SD card to an on-board \gls{eeprom} chip.
This introduced a new constraint, as the \gls{eeprom} chip had only 4 kB of
storage space, as opposed to the near limitless storage capacity of the SD 
card.
We had to reduce the total amount of logging done in order to remain under
the 4 kB limit.
The logging format will remain the same.

\subsubsection{Video Camera}
When this document was first drafted, the video was originally designed to be
handled by the \gls{obc}.
However, as the electrical system was designed and finalized, it became easier
to have the video completely handled in hardware.
As a result, the \gls{obc} does not interact with the video camera beyond
turning it on at the beginning of the experiment and turning it off at the end.
The electrical system manages storing the data.

\subsubsection{Data Visualization and Processing}
The only change to the design of the data visualization and processing
component is the parsing component for the log file. 
Instead of opening an \gls{ascii} text file stored on the SD card, this
component must parse a raw binary file copied from the \gls{eeprom} chip.
