\subsubsection{Technical Information}
I gained a significant amount of technical knowledge over the course of the year. Most of this 
is related to pathing and automation and configuration spaces since I did not have previous 
experience with that. I gained a significant amount of the technical information that I needed 
on configuration spaces from Dr. Smart and from books that I got from the Valley Library. I 
learned how to handle 4D arrays and how to map real space to a configuration space. I learned 
about Dijkstra's Algorithm versus A* and how to implement all this on a system with low computing
 resources. I also encountered the interesting problem of how to test all this on Earth under the
 influence of gravity since the arm's ability to move is dependent on it being in microgravity 
conditions.

\subsubsection{Non-Technical Information}
Over the course of the year, but especially in the last month with Expo, I developed some 
non-technical skills. Most of this has to do with navigating the logistics of the project such as
 fundraising, communicating with the OSU Foundation, and communicating the project to a wide and 
varied audience. Expo allowed me to learn how to talk about this project in a way that can appeal
 to a variety of people with different technical or non-technical backgrounds. Explaining the 
method by which we accomplished four degree of freedom motion in space was not the easiest task, 
but throughout the day I was able to refine my explanation to match the level of the audience. I 
was surprised to find that some of my most technical conversations were with the youngest people,
 as some high school girls talked through all of the code with me and asked about machine 
learning. I was very happy to talk to them, especially when Amber and I found out that one of 
them was going into an ASE internship, the same program that both of us went through in high 
school that got us into the fields that we are in today. The other main type of communication 
that I engaged in for this project was to discuss funding opportunities with the OSU Foundation. 
I was able to have a good conversation with a fundraiser for the foundation thanks to an 
introduction from my mother. While he was not able to help us secure grant money due to the time 
limitation, he was able to put me in contact with someone to set up an OSU Crowdfunding campaign. He was also willing to help us find grant money, and even found us a possible grant to apply 
for, even though we ended up being ineligible for it. Starting this relationship may be helpful 
to future RockSat-X projects at OSU.

\subsubsection{Project Work Information}
I developed some project work skills through this project. As I mention in my Team Work section, 
unique approaches to project work is characteristic of my personality type and is something that 
I struggle with as well. Now that I am aware of this characteristic, I can work to change my 
behavior so that it does not negatively impact my future work. My tendency when I'm working is 
to do things in bursts during which I work very hard without coming up for air and then avoiding 
the project for a little while before repeating the process. While this is great for creative 
and technical development, it can be stressful for the people depending on me, which is not 
acceptable. As such, I need to be careful in the future to communicate what I am doing when and 
to stop worrying that everyone is judging me harshly for only working at odd times. That is a 
silly anxiety to let get in the way of good communication. I learned a lot about how I do 
project work and I have some ideas on how to further improve, but I think we all ended the term 
on a strong note for getting things done.

\subsubsection{Project Management Information}
This year I mainly developed existing project management skills. I was able to quickly and 
painlessly get our github repository up and running thanks to past experience setting up version 
control systems for new organizations. I was also able to run the Pathing and Automation 
Cross-Functional team meetings with Mechanical and Electrical Engineers. This improved 
communication between the majors and gave me a support structure for completing the technical 
portion of the project that I am responsible for.

\subsubsection{Team Work Information}
I learned more about team work and how I work with others over the course of this year. We had a 
couple of disputes within the team. We were always eventually able to resolve these conflicts, 
but from this situation I learned that open communication and good listening can help to avoid 
a lot of these conflicts in the first place, or help to resolve them quickly and painlessly. 
I also learned that being reliable and communicating progress is critical to group success. I 
also learned from the personality tests that we took for the peer review assignment that those 
are two things that my personality type is notoriously bad at. Those are two skills that I was 
able to improve over the course of the term, but that I should continue to be aware of so that 
they do not negatively impact my career. My greatest regret of the term is ever letting my 
teammates down. But 
I did not just learn about handling negative team dynamics. I also learned how to foster a good 
team dynamic. I was able to bring cold brew coffee for my teammates on one of the days that we 
had class, which made us all feel a little more like a family, sipping our matching jars of 
coffee at 8am when we all would usually be asleep. I also learned about the power of delegation 
and asking for help, two things at which I know I am deficient. However, in being aware of this 
challenge, I was able to make the conscious choice to delegate some tasks and ask for help when 
I got stuck. There was still room for improvement, but it was a good start. Finally, I further 
developed my collaborative abilities. I really enjoyed working with my team this year, and I 
think that we produced a quality project and quality reports and documentation. I think a lot of 
this was due to the efficient and collaborative way that we worked together to get it all done.

\subsubsection{If you could do it all over what would you do differently?}
If I could redo this whole project and the whole year, I would want to get the Configuration Space done way sooner. If we had talked to Dr. Smart earlier on in Fall Term and found the motors that we needed before Winter Term, we would have been able to get a lot more done. However knowing who to talk to and what hardware we needed was a challenge of its own. It would have been helpful to start the fundraising earlier as well. If I had been introduced to my contact at the OSU Foundation earlier in the year, he might have been able to help us apply for more grants. But once again that is not something that I could change.

One technical change that I would make is to increase the power of the \gls{obc} as that is a serious limiting factor for our ability to perform the pathing computations on orbit, or even storing a full, detailed configuration space. Our choice of \gls{obc} was somewhat limited by budget. In the future, further fundraising or support from the school for missions could alleviate this concern.

