\subsubsection{Week 3}
\subsubsubsection{Helena Bales}
\textbf{Progress} \\
This week I made significant strides in the design of our project. I wrote part of the Project Definition assignment. I started the Project Description with a description of the problem, broken down into the requirements of the RockSat-X program and the payload that we decided on for the project. In order for our senior design project to be successful, we have to build the payload, meet the RockSat-X project requirements (such as testing, documentation, and design reviews), and meet the capstone class requirements. Our payload idea is a mechanical arm, and as a project it is capable of meeting all the requirements.

While the Project Definition document met our capstone class requirements for the week, there were also RockSat-X requirements to be met this week. The RockSat-X CoDR (Conceptual Design Review) was this week. As a large group (including two teams of ME's, one team of EE's, and the CS team) we developed the CoDR powerpoint that was presented yesterday to RockSat-X. This document included all of our conceptual payload designs thus far, and was our first time presenting our designs to the RockSat-X group. Following that presentation, in order to meet the RockSat-X requirements, we took a group photo.

In addition to the RockSat-X requirements and the capstone class requirements, we met the payload requirements by meeting with Nancy Squires to discuss the project, get approval of the Project Definition assignment, and discuss starting an official Space Lab at OSU.

\textbf{Plans} \\
The next week will be focusing on the development of documents for Senior Design class as well as for the RockSat-X project. Specifically, we will be revising the Project Description document and begin the Requirements Document. We will also be continuing the design process for the payload with the other teams.

\textbf{Problems} \\
The problems that we have encountered have been minimal so far.

\subsubsubsection{Amber Horvath}
\textbf{Progress} \\
This previous week, our team made great strides by writing up our problem statement, which served useful in nailing
 down what we hope to accomplish this coming term, and how we can actualize these goals. We also took a group photo 
 with our full team (comprised of us three computer scientists, six mechanical engineers, and three electrical 
 engineers). This photo was required by Dr. Nancy Squires, our advisor. We also presented our plans in the form of a 
 powerpoint to our Colorado collaborators over a group call.

\textbf{Plans} \\
For the next week, we are going to meet with our team to discuss future prospects for our project. Since we've chosen 
a design, our next step is to begin prototyping implementation designs and begin research on the telemetry aspect of 
the project. We are also going to use our RockSat-X project as a stepping stone to introducing a designated lab for 
satellite creation to Oregon State University. In order to do that, however, we will need to work on finding space on
 campus for such a lab and lobbying for its creation. While this might seem outside of the scope of the project, it 
 would be greatly beneficial for future students who have an interest in space exploration and development.

\textbf{Problems} \\
So far, we have not encountered any barriers. The only issue is coordinating schedules and planning with 11 other 
students. Finding times where everyone is available is difficult, but since we have an effective group chat and a
 Google Calendar with everyone's individual schedules filled in, we've been doing our best to mitigate this issue.
\subsubsubsection{Michael Humphrey}
\textbf{Progress} \\
This past week the Hephaestus project team accomplished several important milestones. We completed our first presentation to the RockSat-X organizers and took a group picture to start raising funding. We are also starting to narrow down our design for the final \gls{payload}.

\textbf{Plans} \\
Because the mechanical and electrical design of the project is not yet finalized, the software team has not yet had any important responsibilities. The electrical team is forbidden from using a device like a Raspberry Pi or an Arduino, so they have decided to use an AVR microcontroller. Amber and I have not used one of these devices, although Helena has. Amber and I will need to start doing research on programming for these devices. We will be using C to program the microcontroller. We won't be able to write any code until the electrical design (i.e. inputs and outputs) are finalized, but we can start creating a software design of how we want the software to work.

\textbf{Problems} \\
No problems have been encountered yet.

\subsubsection{Week 4}
\subsubsubsection{Helena Bales}
\textbf{Progress} \\
This week I was at the Grace Hopper Celebration of Women in Computing. I did not do any work directly on the RockSat-X project, but I did talk to many people about Computer Science and space exploration.

\textbf{Plans} \\
Next week will be focusing on the development of the requirements document for Senior Design and PDR presentation. The PDR presentation is coming up and will require us to compile a powerpoint about our design, practice presenting it, and presenting it for the RockSat-X program.

\textbf{Problems} \\
I encountered a significant obstacle to completing work this week. I did not have internet access at Grace Hopper, so I was unable to work on the project or create an update.

\subsubsubsection{Amber Horvath}
\textbf{Progress} \\
This week, our team worked to figure out more logistical details. We assembled two new sub-teams, one is the budget 
planning committee (to which I represent the CS group), and a group to prepare a slideshow presentation for an event 
happening next week. As for progress on the tool, we just recruited a new post-grad to help who has a degree in 
mechanical engineering but is coming back to school for a computer science degree. He is currently enrolled in CS 161 
and thus has experience in C++ and nothing else. 

\textbf{Plans} \\
Our next initiative is to meet up with the electrical engineers to figure out constraints between hardware and 
software and read up on some of the documentation on what we can load into our payload as far as the robotics side 
goes.

\textbf{Problems} \\
Our barriers have been limited to task and time managing. Between the amount of students, the amount of deadlines, and 
the differing levels of expertise in each others domains, it can be hard to keep communication clear and comprehensive.
Meeting times have also been difficult to coordinate due to the different schedules.

\subsubsubsection{Michael Humphrey}
\textbf{Progress} \\
Similar to week 3's blog post, this past week the Hephaestus Software Team did not have any major responsibilities. We attended the Hephaestus team meetings where the mechanical and electrical designs are still being worked out. We are going to have more communication with the Electrical Engineering team to determine the computing platform and computation restrictions. We also began working out budget numbers.

\textbf{Plans} \\
This next week we will be creating several presentations. I will be partly responsible for a 6 minute 40 second presentation to compete for a \$1,000 cash prize. Other fundraising efforts are also in progress. We will also be meeting with the Colorado Space Grant committee for our next presentation for them. We will also need to start working on revising our Problem Statement and start drafting our Requirements document and any other documentation we need.

\textbf{Problems} \\
Currently, the software team is blocked by the electrical team. Until they finalize a design, we cannot start coding. We will be in communication with them, however, to determine what considerations they need to take for the design.

\subsubsection{Week 5}
\subsubsubsection{Helena Bales}
\textbf{Progress} \\
This week was focused on developing the requirements document for Hephaestus and revising the Project Description document. The revision of the Project Description document was turned in on Wednesday after adding more of a focus on the software side of the project. The first draft of the requirements document will be turned in by the end of the day today. I focused on creating the outline of the document and writing the introduction. The introduction establishes a purpose and description of the document, an overview of the mission description, the mission success criteria, and the priorities for the requirements. The rest of the document describes the functional and non functional requirements that we have established for the software that controls the Hephaestus payload.

\textbf{Plans} \\
The next week will focus on creating a solid final draft of the Requirements Document and presenting PDR. That will require meeting as a group to practice presenting PDR and meeting as a group to present PDR.

\textbf{Problems} \\
Availability has been a problem this week. It has been a challenge to fit all of the large group meetings into my schedule and still have time to catch up on homework after Grace Hopper and work.

\subsubsubsection{Amber Horvath}
\textbf{Progress} \\
This week we refined our problem statement and created a rough draft of the Requirements document. I was unavailable 
to attend our weekly meeting, but I did attend the first budget committee meeting. The meeting was held over the app 
Discord, and we came up with a budget for the rest of our project's lifespan.

\textbf{Plans} \\
Our next initiatve is to gather information from the electronics team to gather information for our requirements
document. This is important, as hardware limitations will greatly impact our software design choices.

\textbf{Problems} \\
The problems we have encountered include illness and traveling. When we do not have teammates available, it can be
difficult to be as productive due to reduced man power. Hopefully everyone comes back happy and healthy soon.
\subsubsubsection{Michael Humphrey}
\textbf{Progress} \\
Since our mechanical and electrical design is still in progress, we have made no progress in the past week toward writing any software. Only work done was finishing the problem statement assignment and drafting our requirements document.

\textbf{Plans} \\
For the next week we will be getting datasheets and other information from the electrical team to aid in drafting our requirements documents. Any limitations of the hardware will be taken into consideration for the software requirements. Those materials should be made available by the electrical team by early next week.

\textbf{Problems} \\
Problems encountered this week were mostly personnel issues. Some of our team has been on vacation and one member is now sick and unable to make it on campus at all. I feel myself coming down with my second illness this term, which will make it even more difficult to get the required signatures we need.

\subsubsection{Week 6}
\subsubsubsection{Helena Bales}
\textbf{Progress} \\
This week, work focused on the development of the Requirements Document for Senior Design and finalizing the PDR presentation for RockSat-X. I mainly focused on the Requirements Document, and did significant work on the structure and content of that document. We turned in a draft first, then flushed it out to a final document that was turned in on Friday of Week 6. I focused on the functional requirements, introduction, and structure of the paper. For the PDR presentation, we had to develop requirements and a plan to meet the requirements. There was a lot of overlap in content between PDR and the Requirements Documents, which was ideal for finishing both of these big documents in the same week. In preparation for this presentation, we had one meeting where we all went over content and one where we practiced the presentation. The final presentation for PDR (Preliminary Design Review) was at 7am on Thursday of week 6. Finally, I revised the README for this repository, so that it was more informative regarding the structure, contents, and context of this repository.

\textbf{Plans} \\
Next week will focus on finalizing major design choices and developing the technical review. The design choices that need to be finalized include the method for assigning test points and the operational modes of the arm.

\textbf{Problems} \\
None.

\subsubsubsection{Amber Horvath}
\textbf{Progress} \\
This week, our team worked on finalizing our Requirements Document, along with doing our first Product Design Review 
meeting with our collaborators in Colorado. We also moved forward on some key design choices, such as choosing to 
program in C on an ATMega128 microcontroller. We also worked on communication between our teams by bringing in Discord 
(the app) as a new medium of communication.

\textbf{Plans} \\
Next week, we hope to work on the framework for the software. With a key design choice in place, we can make progress
towards actual implementation. This is an exciting turning point in our project!

\textbf{Problems} \\
Problems included communication outside of our sub-team. It can be hard to communicate within the larger group.

\subsubsubsection{Michael Humphrey}
\textbf{Progress} \\
This week was spent finalizing our software requirements for the project. We did extensive research into the details of the mechanical and electrical design of our \gls{payload} and drew up documents with specifics such as coordinate systems and \gls{payload} layout. We now have a basis for creating our software.

\textbf{Plans} \\
For the next week, I believe we will be able to start writing the framework for the \gls{payload}. We probably won't be able to start programming the actual function of the \gls{payload} until it is built, but we can create the structure of how our software will be laid out.

\textbf{Problems} \\
Some problems were encountered this week with communication outside of our sub-team, but those have been resolved and shouldn't occur again in the future.

\subsubsection{Week 7}
\subsubsubsection{Helena Bales}
\textbf{Progress} \\
This week we are developing the Technical Review Document for Senior Design. As such, we have divided the requirements up between the three of us as follows:

\textbf{Amber Horvath:}

    Emergency Expelling of Payload

    Program Modes of Operation

    Target Success Sensors

\textbf{Helena Bales:}

    Target Generation

    Movement of arm

    Arm position tracking

\textbf{Michael Humphrey:}

    Telemetry

    Video Camera

    Data visualization and processing

Each of us shall be responsible for insuring the completion of their assigned tasks. We will focus on our assigned tasks for the tech review. This week has focused on defining and assigning the requirements to each of us. We have also finalized some design choices, specifically in the modes of operations, emergency procedures, and arm target generation.

\textbf{Plans} \\
Next week we will complete and turn in the tech review on Monday of Week 8. Before that date we will be finishing that document. After the completion of the tech review we will be going back through past documents and including all suggestions we have received as feedback throughout the course. We will be doing this to prepare for the final document to be turned in on December 4th. We will also be preparing our designs and requirements for our big RockSat-X review during weeks 10 or 11.

\textbf{Problems} \\
We mainly are encountering the issue that we have too many assignments due on or before Monday of Week 8.

\subsubsubsection{Amber Horvath}
\textbf{Progress} \\
This week, our team worked on finalizing our Tech Review and continued collaboration with the larger RockSat-X team. 
Together, we're working to reach conclusions on how the design will be so we can hopefully dive into implementation 
within the next couple weeks. We've settled upon most choices, but it is difficult to reach conclusions with the rest 
of the team about where we stand and how everything will fit together. We did divide work between the three of us
for the software components that are necessary to the overall design of the project. 

\textbf{Plans} \\
Next week I will be out of town for most of the week, so I am leaving most decision making to the rest of my team to 
figure out as far as moving forwards.

\textbf{Problems} \\
Our main issues include traveling as that complicates our established workflow and coordinating all of our deadlines,
both for this class and other classes.

\subsubsubsection{Michael Humphrey}
\textbf{Progress} \\
Last week we developed the requirements of our system a lot. We explored technologies that we want to use and confirmed many details with the robotics and electronics team about the requirements of the \gls{payload}. For me, last week was spent primarily going to meetings, relaying information to teammates, and doing research into potential technologies we can use.

\textbf{Plans} \\
Next week, we will hopefully begin implementation of the software. I need to set up a meeting with the electrical team. I can't remember what they want to talk about but we definitely meet as a team with them. Most likely all of the CS team won't be able to make it, and this is a challenge we will need to overcome.

\textbf{Problems} \\
Problems I encountered included finding adequate solutions for the telemetry technology. I thought it would be easy to find several solutions we could use, but it turns out that most of the solutions I found were not compatible with our system for one reason or another. Mostly because none of them actually dealt with the transmission of the data itself, but what it did with the telemetry after it was collected. Other reasons were that they were implemented in the wrong language.

\subsubsection{Week 8}
\subsubsubsection{Helena Bales}
\textbf{Progress} \\
This week we finalized and turned in the technical review document. Preparing this document required meeting as a group to talk about potential solutions, then documenting the solutions that we came up with. This week we also talked to the Electrical Engineering group to make sure that our plans were consistent and that we would be able to work together on the software/hardware interface in the future.

\textbf{Plans} \\
Next week we plan on starting the Design Document and the presentation for the end of the term and our CDR presentation with RockSat-X.

\textbf{Problems} \\
I have been experiencing technical issues with my computers, so that is something that I will need to resolve before I can seriously start working on the Design Document.

\subsubsubsection{Amber Horvath}
\textbf{Progress} \\
This week, we finalized our Tech Review. Most of the week I was out of town, but we are setting up our relations with 
the electrical team for next term and figuring out meeting times for our larger group next term.

\textbf{Plans} \\
Next week we hope to meet with the electrical team, but this may be difficult to accomplish as it is Thanksgiving
and a lot of people will be heading out of town to meet with family.

\textbf{Problems} \\
Issues encountered included figuring out specifics regarding the implementation efforts that will need to be made.
Looking into potential solutions and having to deal with some of the ambiguities about the hardware solutions
presented so far made figuring out the software solutions difficult. However, as more information is
presented about hardware solutions, this issue should resolve itself.

\subsubsubsection{Michael Humphrey}
\textbf{Progress} \\
Last week I helped start our Design Document. We've created the structure for the document and pasted the relevant sections from our previous documents. I set up meetings with the Electrical team and started communication with them to nail down specific software communication requirements. They're going to create a sort of "firmware" for the \gls{payload}, meaning they'll write the code that interacts directly with the hardware, and they'll expose an abstract interface for the Software team so we only need to call something like moveArm(x, y, z); to control the movement of the arm.

\textbf{Plans} \\
Next week we need to finalize the details of how we want to control the \gls{payload} arm. This will probably mean writing an API that we want the Electrical team to implement. We also need to prepare for the CDR coming up in a couple of weeks. This means we need to fill out the slides the Software team is responsible for. There will probably be other work for this presentation that we will tackle as it comes up.

\textbf{Problems} \\
No problems this week.

\subsubsection{Week 9}
\subsubsubsection{Helena Bales}
\textbf{Progress} \\ 
This week we started the Design Document and slides for the presentation for this class and for our RockSat-X program CDR. We met in order to discuss the solutions that we wanted to choose for each of the requirements. During that meeting I updated our Design Document to reflect the choices that we made, and created issues to reflect the tasks that we have yet to complete.

\textbf{Plans} \\ 
In the next week we will be finishing the Design Document, finishing our slides for the class presentation, finishing our slides for the CDR presentation, practicing the CDR presentation, and starting to compile the progress update assignment.

\textbf{Problems} \\ 
I am still experiencing technical issues with my computer, but less seriously than before, so progress has been made there.

\subsubsubsection{Amber Horvath}
\textbf{Progress} \\
This week was a bit of an odd week due to the Thanksgiving holiday. However, despite the holidays, our team still met 
up to discuss planning how we will finish the term. There still lies some action items that must be completed such as 
resolving issues on our Tech Review document, preparing for the Critical Design Review, and doing our end of term 
project assigned for this class. We will be meeting on Tuesday of this next week to further our plans and see how 
much we have accomplished over the break.

\textbf{Plans} \\
Next week will include meeting to discuss where we stand as a group and what work needs to be completed by what date.
As the end of the term approaches, everyone is more busy and juggling many assignment deadlines from other classes
so touching bases to ensure that the work required for this class gets done.

\textbf{Problems} \\
The main issue this week was scheduling conflicts due to Thanksgiving and the aforementioned deadlines drawing
close. Staying on top of work is sometimes harder than one would imagine.

\subsubsubsection{Michael Humphrey}
\textbf{Progress} \\
This week I didn't do much. The Software team has created an outline for our design document but I haven't added my parts in. I don't foresee it being too much work, as it's mostly already written from the tech review. More details just need to be added. Due to it being Thanksgiving week, I have delayed working on classwork in favor of helping my family prepare for Thanksgiving.

\textbf{Plans} \\
Next week we need to finish our rough draft of the design document as well as write an outline for our presentation.

\textbf{Problems} \\
No problems were encountered this week.

\subsubsection{Week 10}
\subsubsubsection{Helena Bales}
\textbf{Progress} \\ 
This week we finished up the Design Document and started the Progress Update write up and presentation. We also prepared for CDR by adding slides to the presentation. In order to finish the design document we talked about how to solve each of the issues from the Requirements document. Once we picked a solution to pursue, we each added detail to our solutions. The CDR presentation was adapted to form the start of our Progress Update presentation since it already describes the project and our work thus far.

\textbf{Plans} \\ 
Next week we will be finishing our progress update write up and presentation. We will do the write up first, then make sure that the presentation slides cover the content from the write up, and finally record the presentation.

\textbf{Problems} \\ 
None.

\subsubsubsection{Amber Horvath}
\textbf{Progress} \\ 
This week our Project Design Document was due. Due to the business of Thanksgiving and other classes interfering, 
it was rather difficult getting this document done and over with. However, we managed to pull through. We also
made progress on getting ready for next term with the larger senior design group. We came up with a meeting time
for next term and made plans to meet up with the electrical engineers once more to work out specifics
of our design choices.

\textbf{Plans} \\ 
Next week we will need to submit our progress report (which will include entries such as this one) and do
our Critical Design presentation for the people in Colorado. Our team will most likely meet to work on the
project over winter break, but specific times for that have not been chosen yet.

\textbf{Problems} \\ 
The main issues this week included miscommunications regarding work balance and meeting times. Some
members were unable to attend previously scheduled meetings, resulting in work not being distributed
properly and a lot of questions were fielded online as opposed to in person, which slowed our workflow.
However, all work was eventually submitted on time, but it was a more difficult road to complete this then
what we hope to achieve in the future.

\subsubsubsection{Michael Humphrey}
\textbf{Progress} \\
This week we made a lot of progress finalizing the design for the \gls{payload} software. This was mostly a result of writing the design document. There was much communication with the electrical team.

\textbf{Plans} \\
Next week is finals week. We will be writing our progress report and recording our presentation.

\textbf{Problems} \\
One problem we are encountered is the slow response to questions that arise about the RockSat-X program. I have several questions about the format and delivery of telemetry data that won't be answered until mid- to late-next week. That information was not able to be included in the design document.
