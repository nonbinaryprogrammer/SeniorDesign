\documentclass[letterpaper,10pt]{article}

\usepackage{geometry}
\usepackage{hyperref}
\usepackage[nopostdot]{glossaries}
\usepackage[pdftex]{graphicx}
\usepackage{tikz}
\usepackage{wrapfig}
\geometry{textheight=8.5in, textwidth=6in}
\newenvironment{bottompar}{\par\vspace*{\fill}}{\clearpage}

\makeglossaries
\loadglsentries[main]{Glossary}

\title{Software Subsystem Test Design For RockSat-X Payload - Hephaestus}
\author{Helena~Bales\\ \\ CS461 - Fall 2016}

\parindent = 0.0 in
\parskip = 0.1 in

\begin{document}
\maketitle

\begin{abstract}
The \gls{osu} RockSat-X payload Hephaestus is a proof of concept for the 
construction of physical structures in space using a robotic arm. 
This document shall describe the experiments that will be used to test three of 
the functional requirements of the Hephaestus payload described in previous 
documents and reviewed briefly in this document.

The purpose of these experiements is to discover bugs in the software prior to 
system integration into the RockSat-X rocket. The experiments shall be performed 
throughout the implementation and integration phases of the payload development. 
The experiments shall constitute one third of the module tests and shall cover 
the requirements for target generation, arm movement, and arm position tracking.
\end{abstract}

\begin{bottompar}
Approved By
\_\_\_\_\_\_\_\_\_\_\_\_\_\_\_\_\_\_\_\_\_\_\_\_\_\_\_\_\_\_\_\_\_\_\_\_\_\_\_\_\_\_\_\_\_\_\_\_\_\_\_\_\_\_\_\_\_\_\_\_\_\_\_
Date \_\_\_\_\_\_\_\_\_\_\_\_\_\_\_\_\_\_\_\_\_\_\_\_\_\_\_\_ \\


Approved By
\_\_\_\_\_\_\_\_\_\_\_\_\_\_\_\_\_\_\_\_\_\_\_\_\_\_\_\_\_\_\_\_\_\_\_\_\_\_\_\_\_\_\_\_\_\_\_\_\_\_\_\_\_\_\_\_\_\_\_\_\_\_\_
Date \_\_\_\_\_\_\_\_\_\_\_\_\_\_\_\_\_\_\_\_\_\_\_\_\_\_\_\_ \\


Approved By
\_\_\_\_\_\_\_\_\_\_\_\_\_\_\_\_\_\_\_\_\_\_\_\_\_\_\_\_\_\_\_\_\_\_\_\_\_\_\_\_\_\_\_\_\_\_\_\_\_\_\_\_\_\_\_\_\_\_\_\_\_\_\_
Date \_\_\_\_\_\_\_\_\_\_\_\_\_\_\_\_\_\_\_\_\_\_\_\_\_\_\_\_ \\


Approved By
\_\_\_\_\_\_\_\_\_\_\_\_\_\_\_\_\_\_\_\_\_\_\_\_\_\_\_\_\_\_\_\_\_\_\_\_\_\_\_\_\_\_\_\_\_\_\_\_\_\_\_\_\_\_\_\_\_\_\_\_\_\_\_
Date \_\_\_\_\_\_\_\_\_\_\_\_\_\_\_\_\_\_\_\_\_\_\_\_\_\_\_\_ \\
\end{bottompar}

\clearpage
\tableofcontents
\clearpage

\section{Introduction}
This document is an initial design of the experiments that will be conducted in 
order the test the modules of the Hephaestus payload. The purpose of performing 
these tests is to discover bugs in the Software Subsystem prior to integration 
of the software with the hardware and electrical subsystems in the Hephaestus 
payload and to the integration of the payload with the 2017 RockSat-X rocket. 

These experiments will constitute the module tests to be performed throughout 
the implementation and integration phases of the Hephaestus project. Data shall 
be collected from these experiements, which will provide a guide for fixing bugs 
and insuring that the payload operates within our desired parameters. 

System and integration tests shall be performed to supplement these unit tests. 
In addition to these unit tests, unit tests shall be developed and performed for 
the six other functional requirements defined in the Technical Review Document.

\subsection{Document Overview}
This document includes an overview of the three functional requirements that
will be tested. These requirements include target generation, arm movements, and 
arm position tracking. The requirements were defined in the Technical Review 
and Requirements Overview documents. They will be covered again in this
document. Additionally, this document will describe the experiments that will be 
performed as tests. These descriptions will include the experiment's purpose, 
pre-conditions, post-conditions, tools required, method, and data. Finally, this 
document will include the inputs that will be used as test cases for the 
experiments.

\section{Requirements Review}

\subsection{Target Generation}
\subsubsection{Description}
Target points shall be generated in 3D Polar form. These points shall include an 
angle, radius, and height. The angle shall be measured from the normal vector, 
defined as the direction of payload deployment from the rocket body. The radius 
shall be measured in centimeters from the center of the mount point of the 
Hephaestus arm to the arm assembly body. The height shall be measured in 
centimeters above the arm assembly body platform.

The target points shall be the points that the Hephaestus arm attempts to touch 
during the experiment. These points will constitute the total on-orbit test of
the Hephaestus arm and shall therefore be representative of the range of motion
of the Hephaestus arm.

\subsubsection{Requirements to be Tested}
\begin{enumerate}
\item{The targets generated are within the range of motion of the arm.}
\item{The targets are generated in 3D polar form.}
\item{The set of target generated is representative of the arm's abilities.}
\end{enumerate}

\subsection{Arm Position Tracking}
\subsubsection{Description}
The position of the arm shall be tracked and stored in 3D Polar Coordinates, as
described above. The position of the arm following a motion will be calculated
using the initial position of the arm and reverse kinematics. This means that
the position will be calculated using the motion of the motors.

The position of the arm shall be reset periodically in order to limity the 
propagation of error over the course of the flight. This will be accomplished by
returning to the initial position and resetting the position values to zero. The
frequency with which the position is reset shall be determined by experiment.
The position of the arm shall be denoted \(p\) and shall represent the location
of the tip of the arm.

\subsubsection{Requirements to be Tested}
\begin{enumerate}
\item{Determine the rate of deterioration of position accuracy as a function of
time, number of operations performed, and distance traveled.}
\item{Determine the accuracy of the arm's location at a point.}
\item{Determine the accuracy of the calculation of the arm's current location by
reverse kinematics.}
\end{enumerate}

\subsection{Arm Movement}
\subsubsection{Description}
The motions performed by the arm shall constitute the arm's functionality. The
movement of the arm shall be determined by finding the most efficient path from
\(p_{current}\) and \(p_{target}\). The efficiency of the path shall be
determined by minimizing the rotation of the motors. The position of the arm
shall be tracked as described in the previous subsection.
\subsubsection{Requirements to be Tested}
\begin{enumerate}
\item{The path generated is the most efficient (as defined above) path.}
\item{The arm follows the path defined in (1).}
\item{The arm accurately moves to the target point.}
\end{enumerate}

\section{Tests}

\subsection{Experiment 1: Accuracy of Stored Position at a Point}
\subsubsection{Purpose}
The purpose of this experiment is to test the accuracy of the position of the
arm at a point. The arm's position is stored in 3D polar form and calculated
using reverse kinematics. This experiment will show the accuracy of the arm as
it executes its test motions. This will be accomplished by measuring the actual
poisition of the arm and comparing it to the target position of the arm.

\subsubsection{Pre-Conditions}
The arm is at a location that is known. The arm has a target location towards
which it will move.

\subsubsection{Post-Conditions}
The arm is at a new location that is approximately known. The exact location of
the arm is measured. The expected location of the arm is known.

\subsubsection{Tools}
\begin{itemize}
\item{Hephaestus payload}
\item{Protractor}
\item{Ruler}
\item{Right angle or Level}
\end{itemize}

\subsubsection{Method}
\begin{enumerate}
\item{Move the arm to the position given by the next target in the test plan.}
\item{Store that target position as \(p_{target n}\).}
\item{}
\item{}
\end{enumerate}

\subsubsection{Data}
\subsubsection{Sources of Error}

\subsection{Experiment 2: Deterioration of Position Accuracy Over Course of Flight}
\subsubsection{Purpose}
\subsubsection{Pre-Conditions}
\subsubsection{Post-Conditions}
\subsubsection{Tools}
\subsubsection{Method}
\subsubsection{Data}
\subsubsection{Sources of Error}

\subsection{Experiment 3: Validation of Path Efficiency}
\subsubsection{Purpose}
\subsubsection{Pre-Conditions}
\subsubsection{Post-Conditions}
\subsubsection{Tools}
\subsubsection{Method}
\subsubsection{Data}
\subsubsection{Sources of Error}

\subsection{Experiment 4: }
\subsubsection{Purpose}
\subsubsection{Pre-Conditions}
\subsubsection{Post-Conditions}
\subsubsection{Tools}
\subsubsection{Method}
\subsubsection{Data}
\subsubsection{Sources of Error}

\section{Test Inputs}
\subsection{Input Requirements}
\subsection{Test Input Sets}

\section{Conclusion}
\end{document}
