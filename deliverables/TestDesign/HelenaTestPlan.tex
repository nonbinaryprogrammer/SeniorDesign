\documentclass[letterpaper,10pt]{article}

\usepackage{geometry}
\usepackage{hyperref}
\usepackage[nopostdot]{glossaries}
\usepackage[pdftex]{graphicx}
\usepackage{tikz}
\usepackage{wrapfig}
\geometry{textheight=8.5in, textwidth=6in}
\newenvironment{bottompar}{\par\vspace*{\fill}}{\clearpage}

\makeglossaries
\loadglsentries[main]{Glossary}

\title{Software Subsystem Test Design For RockSat-X Payload - Hephaestus}
\author{Helena~Bales\\ \\ CS461 - Fall 2016}

\parindent = 0.0 in
\parskip = 0.1 in

\begin{document}
\maketitle

\begin{abstract}
The \gls{osu} RockSat-X payload Hephaestus is a proof of concept for the 
construction of physical structures in space using a robotic arm. 
This document shall describe the experiments that will be used to test three of 
the functional requirements of the Hephaestus payload described in previous 
documents and reviewed briefly in this document.

The purpose of these experiements is to discover bugs in the software prior to 
system integration into the RockSat-X rocket. The experiments shall be performed 
throughout the implementation and integration phases of the payload development. 
The experiments shall constitute one third of the module tests and shall cover 
the requirements for target generation, arm movement, and arm position tracking.
\end{abstract}

\begin{bottompar}
Approved By
\_\_\_\_\_\_\_\_\_\_\_\_\_\_\_\_\_\_\_\_\_\_\_\_\_\_\_\_\_\_\_\_\_\_\_\_\_\_\_\_\_\_\_\_\_\_\_\_\_\_\_\_\_\_\_\_\_\_\_\_\_\_\_
Date \_\_\_\_\_\_\_\_\_\_\_\_\_\_\_\_\_\_\_\_\_\_\_\_\_\_\_\_ \\


Approved By
\_\_\_\_\_\_\_\_\_\_\_\_\_\_\_\_\_\_\_\_\_\_\_\_\_\_\_\_\_\_\_\_\_\_\_\_\_\_\_\_\_\_\_\_\_\_\_\_\_\_\_\_\_\_\_\_\_\_\_\_\_\_\_
Date \_\_\_\_\_\_\_\_\_\_\_\_\_\_\_\_\_\_\_\_\_\_\_\_\_\_\_\_ \\


Approved By
\_\_\_\_\_\_\_\_\_\_\_\_\_\_\_\_\_\_\_\_\_\_\_\_\_\_\_\_\_\_\_\_\_\_\_\_\_\_\_\_\_\_\_\_\_\_\_\_\_\_\_\_\_\_\_\_\_\_\_\_\_\_\_
Date \_\_\_\_\_\_\_\_\_\_\_\_\_\_\_\_\_\_\_\_\_\_\_\_\_\_\_\_ \\


Approved By
\_\_\_\_\_\_\_\_\_\_\_\_\_\_\_\_\_\_\_\_\_\_\_\_\_\_\_\_\_\_\_\_\_\_\_\_\_\_\_\_\_\_\_\_\_\_\_\_\_\_\_\_\_\_\_\_\_\_\_\_\_\_\_
Date \_\_\_\_\_\_\_\_\_\_\_\_\_\_\_\_\_\_\_\_\_\_\_\_\_\_\_\_ \\
\end{bottompar}

\clearpage
\tableofcontents
\clearpage

\section{Introduction}
This document is an initial design of the experiments that will be conducted in 
order the test the modules of the Hephaestus payload. The purpose of performing 
these tests is to discover bugs in the Software Subsystem prior to integration 
of the software with the hardware and electrical subsystems in the Hephaestus 
payload and to the integration of the payload with the 2017 RockSat-X rocket. 

These experiments will constitute the module tests to be performed throughout 
the implementation and integration phases of the Hephaestus project. Data shall 
be collected from these experiements, which will provide a guide for fixing bugs 
and insuring that the payload operates within our desired parameters. 

System and integration tests shall be performed to supplement these unit tests. 
In addition to these unit tests, unit tests shall be developed and performed for 
the six other functional requirements defined in the Technical Review Document.

\subsection{Document Overview}
This document includes an overview of the three functional requirements that
will be tested. These requirements include target generation, arm movements, and 
arm position tracking. The requirements were defined in the Technical Review 
and Requirements Overview documents. They will be covered again in this
document. Additionally, this document will describe the experiments that will be 
performed as tests. These descriptions will include the experiment's purpose, 
pre-conditions, post-conditions, tools required, method, and data. Finally, this 
document will include the inputs that will be used as test cases for the 
experiments.

\section{Requirements Review}

\subsection{Target Generation}
\subsubsection{Description}
\subsubsection{Requirements to be Tested}

\subsection{Arm Position Tracking}
\subsubsection{Description}
\subsubsection{Requirements to be Tested}

\subsection{Arm Movement}
\subsubsection{Description}
\subsubsection{Requirements to be Tested}

\section{Tests}

\subsection{Experiment 1: Accuracy of Stored Position at a Point}
\subsubsection{Purpose}
\subsubsection{Pre-Conditions}
\subsubsection{Post-Conditions}
\subsubsection{Tools}
\subsubsection{Method}
\subsubsection{Data}

\subsection{Experiment 2: Deterioration of Position Accuracy Over Course of Flight}
\subsubsection{Purpose}
\subsubsection{Pre-Conditions}
\subsubsection{Post-Conditions}
\subsubsection{Tools}
\subsubsection{Method}
\subsubsection{Data}

\subsection{Experiment 3: Validation of Path Efficiency}
\subsubsection{Purpose}
\subsubsection{Pre-Conditions}
\subsubsection{Post-Conditions}
\subsubsection{Tools}
\subsubsection{Method}
\subsubsection{Data}

\subsection{Experiment 4: }
\subsubsection{Purpose}
\subsubsection{Pre-Conditions}
\subsubsection{Post-Conditions}
\subsubsection{Tools}
\subsubsection{Method}
\subsubsection{Data}

\section{Test Inputs}
\subsection{Input Requirements}
\subsection{Test Input Sets}

\section{Conclusion}
\end{document}
