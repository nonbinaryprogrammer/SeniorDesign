\subsubsection{Week 1}
\begin{itemize}
\item{
\textbf{Progress}

I made no progress over winter break, other than actually managing to take a vacation. I am very proud of myself.

Since the start of week 1, I met with the Software Team to discuss the schedule that we will have for the term. We decided that we would have an hour long meeting on Mondays after our TA meeting and an hour long meeting with the whole Hephaestus team on Thursdays at 6pm. We have not yet had a Monday meeting because the past two mondays have been cancelled due to weather and MLK Day.

We accomplished some tasks for week 1, which includes adding more content to the System Architecture for the Software Subsystem. In addition to this planning I began development of the test cases that will be used to test the Software Subsystem. Specifically, I am focusing on developing experiments to test the three functional requirements that I was assigned last term. 
}

\item{
\textbf{Plans}

The plan for next week is to finish up the architecture diagram and the test cases and beginning the implementation phase of the project. We need to have a prototype completed and tested by mid February, so we are planning on implementing for three weeks then testing.
}

\item{
\textbf{Problems}

My biggest problem is that I do not have a working computer right now. I have been trying to fix my computer but it has just been a time sink so far. I don't have a solution to this problem.
}
\end{itemize}

\subsubsection{Week 3}
\begin{itemize}
\item{
\textbf{Progress}

This week we started really diving into the motion planning in the Pathing and Automation cross-functional team. I checked out books from the library to help with research into motion planning for robotics, robotics in space, and inverse kinematics.
}
\item{
\textbf{Plans}

Over the next weeks I will be doing research into the issues of path planning and motion tracking on earth and in space.
}
\item{
\textbf{Problems}

I still don't have a working computer with which to do the research.
}
\end{itemize}

\subsubsection{Week 4}
\begin{itemize}
\item{
\textbf{Progress}

This week was continuing research into the pathing and automation portion of the software. I have found that the A* algorithm for pathfinding within a configuration space will be a good solution. Additionally, we will be breaking up the arm into its individual links in order to move the arm to a valid configuration. Essentially, we will start at the base of the arm and move that first, then move up the arm to the next link and move that.
}
\item{
\textbf{Plans}

The next week will be finishing up the research phase for pathing and automation and starting implementations. We will be starting with building the Configuration Space.
}
\item{
\textbf{Problems}

We still haven't figured out a good way to build a C-Space.
}
\end{itemize}

\subsubsection{Week 5}
\begin{itemize}
\item{
\textbf{Progress}

		This week was focusing on figuring out how to build the configuration space (C-Space) in which to perform the path-finding algorithm for the arm's motion. We found that the C-Space need to be in \(R^4\) because we have 4 degrees of freedom in the arm. We also know that for each possible configuration of the arms' motors, we need to know if the configuration is valid in order to map the C-Space. This week we are also starting on the slide for STR.
}
\item{
\textbf{Plans}

The next week will focus on finalizing the slides for STR. Our STR presentation will be at 6am on Friday of week 6. In addition to STR, the Pathing and Automation and Software groups will be meeting with Dr. Smart during week 6 to discuss methods for building the configuration space.
}
\item{
\textbf{Problems}

I am blocked from progressing further with the code for motion planning because we do not yet know enough about the C-Space and how to build it. This should be resolved next week after meeting with Dr. Smart.
}
\end{itemize}

\subsubsection{Week 6}
\begin{itemize}
\item{
\textbf{Progress}

This week involved finishing the presentation for STR, a all-team social event, presenting STR, and meeting with Dr. Smart. The meeting with Dr. Smart was on Monday and provided a lot of useful information for pathfinding and automation. We discussed methods for creating the Configuration Space. Dr. Smart explained the ways in which we should limit the payload to keep the configuration space as a plane in R4. We also discussed the best way to generate the configuration space. The options that we discussed were calculating it mathematically using Inverse Kinematics, running a simulation in solid works, or physically moving the arm to valid configurations and mapping those. Each of these methods has benefits and drawbacks. STR occured on friday. In preparation we created the slides throughout the week. On thursday, at our all-team meeting, we went through all the slides in preparation for the presentation on Friday at 6am. The presentation went very well on Friday. The project reviewer said that she was excited to see our project and that our presentation and progress were both very good. The all-team social was at the All-Team meeting on Thursday after we finished all relevant business. We ordered pizza and played board games. The Software team was divided between the two teams with Michael and I against Amber. Amber's team won the first two rounds, but Michael and I brought in a win in the last round. All in all, it was an effective evening of work and team bonding.
}
\item{
\textbf{Plans}

The next week will focus on creating and recording our presentation for the Senior Design class. We will be working on the presentation on Tuesday, finishing it on Wednesday in order to record the video on Wednesday or Thursday. We will finish the project with editing and posting the video on Thursday and Friday to have it done by Friday. I will also be updating the design documents from last term to reflect the changes we have made. I do not expect there to be significant changes, however there may be some slight modifications to the pathfinding and automation section to reflect what Dr. Smart taught us this week.
}
\item{
\textbf{Problems}

The motors have arrived, so I am no longer blocked on progressing in the code. Following the completion of the presentation for CS462, I will be able to dive into the pathfinding code.
}
\end{itemize}

