\documentclass[letterpaper,10pt]{article}

\usepackage{geometry}
\usepackage{hyperref}
\geometry{textheight=8.5in, textwidth=6in}
\newenvironment{bottompar}{\par\vspace*{\fill}}{\clearpage}

\title{Requirements For RockSat-X Payload - Hephaestus}
\author{Helena~Bales, Amber~Horvath, and Michael~Humphrey\\ \\ CS461 - Fall 2016}

\parindent = 0.0 in
\parskip = 0.1 in

\begin{document}
\maketitle

\begin{abstract}
The Oregon State University RockSat-X team shall be name Hephaestus.
The requirements for the hardware, software, and programmatic development of Hephaestus shall be outlined in the following document.
The mission requires that the payload, an autonomous robotic arm, perform a series of motions to locate predetermined targets.
The hardware shall be capable of performing the motions to reach the targets.
The software shall determine the targets and send the commands to the hardware to execute the motion.
The combination of the hardware controlled by the software shall demonstrate Hephaestus's ability to construct small parts on orbit.
\end{abstract}

\begin{bottompar}
Approved By
\_\_\_\_\_\_\_\_\_\_\_\_\_\_\_\_\_\_\_\_\_\_\_\_\_\_\_\_\_\_\_\_\_\_\_\_\_\_\_\_\_\_\_\_\_\_\_\_\_\_\_\_\_\_\_\_\_\_\_\_\_\_\_
Date \_\_\_\_\_\_\_\_\_\_\_\_\_\_\_\_\_\_\_\_\_\_\_\_\_\_\_\_ \\
Approved By
\_\_\_\_\_\_\_\_\_\_\_\_\_\_\_\_\_\_\_\_\_\_\_\_\_\_\_\_\_\_\_\_\_\_\_\_\_\_\_\_\_\_\_\_\_\_\_\_\_\_\_\_\_\_\_\_\_\_\_\_\_\_\_
Date \_\_\_\_\_\_\_\_\_\_\_\_\_\_\_\_\_\_\_\_\_\_\_\_\_\_\_\_ \\
Approved By
\_\_\_\_\_\_\_\_\_\_\_\_\_\_\_\_\_\_\_\_\_\_\_\_\_\_\_\_\_\_\_\_\_\_\_\_\_\_\_\_\_\_\_\_\_\_\_\_\_\_\_\_\_\_\_\_\_\_\_\_\_\_\_
Date \_\_\_\_\_\_\_\_\_\_\_\_\_\_\_\_\_\_\_\_\_\_\_\_\_\_\_\_ \\
Approved By
\_\_\_\_\_\_\_\_\_\_\_\_\_\_\_\_\_\_\_\_\_\_\_\_\_\_\_\_\_\_\_\_\_\_\_\_\_\_\_\_\_\_\_\_\_\_\_\_\_\_\_\_\_\_\_\_\_\_\_\_\_\_\_
Date \_\_\_\_\_\_\_\_\_\_\_\_\_\_\_\_\_\_\_\_\_\_\_\_\_\_\_\_ \\
\end{bottompar}

\clearpage
\tableofcontents
\clearpage

\section{Introduction}
\subsection{Purpose of Document}
This document shall describe in detail the Hephaestus RockSat-X payload.
It specifies the software behavior of the payload.
This document will not discuss the specific implementations of the hardware or the software.
It will specify the behavior by describing the Functional and Non Functional requirements of the software.
This document will be updated throughout the project.
\subsection{Overview of Document}
This document will first cover the functional requirements of the project, then the non functional requirements.
The Functional Requirements will include descriptions of the main behavior, target generation, movement, operation modes, and telemetry.
Each of these topics will include descriptions of the behavioral requirements for each.
The Non Functional requirements will cover the performance, security, and telemetry.
Each of the non functional topics covered will include the requirements for the quality of each of the topics.
\subsection{Overview of Payload}
The Hephaestus RockSat-X payload shall perform the following operations:
\begin{itemize}
\item{Remain retracted with power off for duration of launch}
\item{Power on at appogee}
\item{Deploy arm assembly body}
\item{Deploy arm}
\item{Perform 360 degree sweep with video camera}
\item{Generate targets for arm motions}
\item{Perform arm motions}
\item{Record each arm motion with video camera}
\item{Retract arm}
\item{Retract arm assembly body}
\item{Power off}
\end{itemize}
\subsection{Mission Success Criteria}
The following criteria determine if the Hephaestus mission will be considered successful post-flight.
The minimum mission success criteria represent the lowest criteria to be met in order for the mission to be considered successful.
If the minimum mission success criteria are not met, then the mission may not be considered successful.
The maximum success criteria define the highest goals for the mission.
Fulfilling any or all of these criteria, in addition to the minimum success criteria, would constitute a highly successful mission.
The success of the mission shall be evaluated by means of video recordings recovered post-flight and telemetry data received during the flight.
\subsubsection{Minimum Mission Success Criteria}
\begin{itemize}
\item{The arm assembly body shall deploy and a video sweep is successfully recorded.}
\item{The arm assembly body shall be fully retracted after data collection.}
\end{itemize}
\subsubsection{Maximum Mission Success Criteria}
\begin{itemize}
\item{The arm assembly body shall deploy and a video sweep is successfully recorded.}
\item{The arm shall make contact with predetermined targets around the payload.}
\item{The camera shall record all instances of contact between the arm and the targets.}
\item{The arm assembly body shall be fully retracted after data collection.}
\end{itemize}
\subsection{Requirements Apportioning}
\subsubsection{Priority 1}
This is the highest priority level. In order for the software system to be considered complete and ready for launch, all requirements of this level must be met.
The completion of only Priority 1 requirements marks the completion of Minimum Mission Success criteria, as defined in section 1.4.
\subsubsection{Priority 2}
Requirements of priority 2 are not required for the release of the software system.
Not completing these requirements must not present a risk to mission success.
The completion of these requirements and successful performance on orbit marks completion of part of the Maximum Success Criteria, as defined in section 1.4.
\subsubsection{Priority 3}
Requirements of Priority 3 are not required for the release of the software system.
Not completing these requirements must not present a risk to mission success.
Completion of all priority 3 requirements and those of higher priority, with successful performance on orbit, marks the completion of the Maximum Mission Success Criteria, as defined in section 1.4.

\section{Functional Requirements}
\subsection{Main Behavior}
\subsection{Target Generation}
\subsection{Movement}
\subsection{Modes}
\subsubsection{Launch}
\subsubsection{Deployment}
\subsubsection{Science}
\subsubsection{Safety}
\subsubsection{Observation}
\subsubsection{Off}
\subsection{Telemetry}

\section{Non Functional Requirements}
\subsection{Performance}
\subsection{Security}
\subsection{Telemetry}


\end{document}
